% =============================================================================
% JCSSE 2026 Conference Poster — EADD
% Explainable Adversarial Drift Detection for MLOps Feature Monitoring
% Portrait A0 poster using beamerposter
% =============================================================================
\documentclass[final]{beamer}
\usepackage[orientation=portrait, size=a0, scale=1.4]{beamerposter}
\usepackage{graphicx}
\usepackage{booktabs}
\usepackage{amsmath,amssymb}
\usepackage{tikz}
\usepackage{multicol}
\usetikzlibrary{arrows.meta, positioning, shapes.geometric, fit, calc}

% ── Colour palette (Mahidol blue + accent) ──────────────────────────────────
\definecolor{MahidolBlue}{RGB}{0,51,102}
\definecolor{MahidolGold}{RGB}{204,153,0}
\definecolor{LightBg}{RGB}{235,241,249}
\definecolor{AccentGreen}{RGB}{34,139,34}
\definecolor{AccentRed}{RGB}{178,34,34}

% ── Beamer theme ─────────────────────────────────────────────────────────────
\setbeamercolor{block title}{fg=white, bg=MahidolBlue}
\setbeamercolor{block body}{fg=black, bg=LightBg}
\setbeamercolor{block alerted title}{fg=white, bg=AccentRed}
\setbeamercolor{block alerted body}{fg=black, bg=LightBg}
\setbeamercolor{block example title}{fg=white, bg=AccentGreen}
\setbeamercolor{block example body}{fg=black, bg=LightBg}
\setbeamertemplate{navigation symbols}{}
\setbeamerfont{block title}{size=\Large, series=\bfseries}

% ── Poster header ────────────────────────────────────────────────────────────
\setbeamertemplate{headline}{
  \leavevmode
  \begin{beamercolorbox}[wd=\paperwidth, ht=12cm]{headline}
    \begin{center}
      \vskip1cm
      {\color{MahidolBlue}\VERYHuge\bfseries Explainable Adversarial Drift Detection\\[0.3em]
       for MLOps Feature Monitoring}\\[1.0cm]
      {\color{black}\LARGE Nusrat Begum\textsuperscript{1} \qquad Thanapon Noraset\textsuperscript{1} \qquad Suppawong Tuarob\textsuperscript{1}}\\[0.4cm]
      {\color{MahidolBlue}\Large \textsuperscript{1}Faculty of Information and Communication Technology, Mahidol University, Thailand}\\[0.3cm]
      {\large\texttt{nusrat.beg@student.mahidol.edu}}
      \vskip0.5cm
    \end{center}
  \end{beamercolorbox}
  \begin{beamercolorbox}[wd=\paperwidth, ht=0.4cm]{separator}
    \color{MahidolGold}\rule{\paperwidth}{0.4cm}
  \end{beamercolorbox}
}

% ── Footer ───────────────────────────────────────────────────────────────────
\setbeamertemplate{footline}{
  \begin{beamercolorbox}[wd=\paperwidth, ht=2cm]{footline}
    \begin{center}
      \vskip0.3cm
      {\color{MahidolBlue}\large JCSSE 2026 --- 23rd International Joint Conference on Computer Science and Software Engineering $\cdot$ June 24--27, 2026 $\cdot$ Bangkok, Thailand}
    \end{center}
  \end{beamercolorbox}
}

\begin{document}

\begin{frame}[t]
\vskip1cm

% ── Three-column layout ─────────────────────────────────────────────────────
\begin{columns}[t]

% ====================== LEFT COLUMN =========================================
\begin{column}{0.31\textwidth}

% --- Motivation ---
\begin{block}{1. Motivation}
\begin{itemize}
  \item ML models in production suffer from \textbf{distribution drift}: input feature distributions shift over time, silently degrading predictions.
  \item Existing unsupervised detectors signal \textit{that} drift occurred but not \textit{which} features drifted or \textit{what} to do about it.
  \item \textbf{Gap:} No unsupervised detector simultaneously provides statistical rigor, feature-level explainability, and automated remediation.
\end{itemize}
\vskip0.5cm
{\color{MahidolBlue}\large\bfseries Research Question:}\\[6pt]
\textit{Can adversarial validation be extended with permutation testing and feature attribution to provide statistically sound, explainable, and actionable drift detection across diverse drift types?}
\end{block}

\vskip1cm

% --- Methodology ---
\begin{block}{2. EADD Framework}

\begin{center}
\begin{tikzpicture}[
    node distance=1.2cm and 0.5cm,
    box/.style={rectangle, draw=MahidolBlue, fill=LightBg, text width=16cm, 
                minimum height=2.4cm, align=center, font=\large,
                rounded corners=6pt, line width=1.5pt},
    arr/.style={-{Stealth[length=8pt]}, line width=2pt, MahidolBlue},
    lbl/.style={font=\large\bfseries, MahidolBlue}
  ]
  \node[box] (step1) {\textbf{Step 1: Data Windowing}\\Reservoir sampling $W_{ref}=500$\\Sliding window $W_{cur}=200$};
  \node[box, below=of step1] (step2) {\textbf{Step 2: Adversarial Classifier}\\LightGBM distinguishes $W_{ref}$ vs $W_{cur}$\\AUC-ROC $>$ 0.7 $\Rightarrow$ suspect drift};
  \node[box, below=of step2] (step3) {\textbf{Step 3: Permutation Test}\\Shuffle labels $B=50$ times\\$p < 0.01 \Rightarrow$ confirm drift};
  \node[box, below=of step3] (step4) {\textbf{Step 4: Attribution \& Prescription}\\TreeSHAP feature ranking\\Classify: univariate / subset / multivariate\\Automated remediation action};

  \draw[arr] (step1) -- (step2);
  \draw[arr] (step2) -- (step3);
  \draw[arr] (step3) -- (step4);
\end{tikzpicture}
\end{center}

\end{block}

\vskip1cm

% --- Key Innovation ---
\begin{exampleblock}{Key Innovation}
\large
EADD is the \textbf{first unsupervised drift detector} that provides:
\begin{enumerate}
  \item \textbf{Statistical rigor} via permutation testing (zero false alarms)
  \item \textbf{Feature-level explainability} via SHAP attribution
  \item \textbf{Actionable prescriptions} mapping drift patterns to MLOps fixes
\end{enumerate}
\end{exampleblock}

\end{column}

% ====================== MIDDLE COLUMN =======================================
\begin{column}{0.31\textwidth}

% --- Experiment 1 ---
\begin{block}{3. Experiment 1 — Temporal Drift Types}
\large
\textbf{Objective:} Can EADD detect all four temporal drift patterns?

\vskip0.5cm
\begin{table}
\centering
\renewcommand{\arraystretch}{1.3}
\begin{tabular}{lcc}
\toprule
\textbf{Drift Type} & \textbf{EADD} & \textbf{D3} \\
\midrule
Abrupt        & \color{AccentGreen}\textbf{5/5} & \color{AccentGreen}\textbf{5/5} \\
Gradual       & \color{AccentGreen}\textbf{5/5} & \color{AccentRed}\textbf{0/5} \\
Incremental   & \color{AccentGreen}\textbf{5/5} & \color{AccentRed}\textbf{0/5} \\
Recurring     & \color{AccentGreen}\textbf{5/5} & \color{AccentGreen}\textbf{5/5} \\
\midrule
\textbf{Overall} & \color{AccentGreen}\textbf{20/20} & \color{AccentRed}\textbf{10/20} \\
\bottomrule
\end{tabular}
\end{table}

\vskip0.3cm
{\color{MahidolBlue}$\Rightarrow$} EADD: \textbf{100\%} detection rate \quad D3: \textbf{50\%} (fails on gradual \& incremental)
\end{block}

\vskip1cm

% --- Experiment 2 ---
\begin{block}{4. Experiment 2 — 13 Real-World Datasets}
\large
\textbf{Objective:} Benchmark against D3 on diverse real-world streams.

\vskip0.5cm
\begin{table}
\centering
\renewcommand{\arraystretch}{1.2}
\begin{tabular}{lrr}
\toprule
\textbf{Dataset} & \textbf{EADD} & \textbf{D3} \\
 & \textbf{MTD} & \textbf{MTD} \\
\midrule
Insects Abrupt       & 173   & 152.5 \\
Insects Gradual      & 9{,}371 & 9{,}481 \\
Insects Incr.\ Abrupt   & \textbf{31}    & 160   \\
Insects Incr.\ Balanced & 102   & 104.5 \\
Insects Reoccurring  & \textbf{131}  & 157   \\
SineClusters         & \textbf{139}  & 229   \\
WaveformDrift2       & \textbf{119}  & 169   \\
\midrule
\textbf{Avg.\ (13 datasets)} & \textbf{1{,}661} & 1{,}725 \\
\bottomrule
\end{tabular}
\end{table}

\vskip0.3cm
{\color{MahidolBlue}$\Rightarrow$} \textbf{0\% missed detection rate} \quad MTD 3.7\% faster than D3
\end{block}

\vskip1cm

% --- Experiment 3 ---
\begin{block}{5. Experiment 3 — Explainability}
\large
\textbf{Objective:} Does SHAP correctly identify which features drifted?

\vskip0.5cm
\begin{table}
\centering
\renewcommand{\arraystretch}{1.3}
\begin{tabular}{lccl}
\toprule
\textbf{Scenario} & \textbf{AUC} & \textbf{Top Feature} & \textbf{Correct?} \\
\midrule
Univariate    & 0.801 & F3 (49.7\%)   & \color{AccentGreen}\textbf{Yes} \\
Subset        & 0.767 & F5, F2, F7       & \color{AccentGreen}\textbf{Yes} \\
Multivariate  & 0.786 & $\leq$14.1\% each & \color{AccentGreen}\textbf{Yes} \\
\bottomrule
\end{tabular}
\end{table}

\vskip0.3cm
{\color{MahidolBlue}$\Rightarrow$} \textbf{100\% feature attribution accuracy} across all three drift patterns
\end{block}

\end{column}

% ====================== RIGHT COLUMN ========================================
\begin{column}{0.31\textwidth}

% --- Experiment 4 ---
\begin{block}{6. Experiment 4 — False Alarm Robustness}
\large
\textbf{Objective:} Do statistical noise patterns cause false alarms?

\vskip0.5cm
\begin{table}
\centering
\renewcommand{\arraystretch}{1.3}
\begin{tabular}{lcc}
\toprule
\textbf{Noise Type} & \textbf{EADD FA} & \textbf{D3 FA} \\
\midrule
Gaussian       & \color{AccentGreen}\textbf{0} & 10.4 \\
Autocorrelated & \color{AccentGreen}\textbf{0} & 87.4 \\
Heteroscedastic & \color{AccentGreen}\textbf{0} & 10.4 \\
Correlated     & \color{AccentGreen}\textbf{0} & 10.0 \\
\midrule
\textbf{Total} & \color{AccentGreen}\textbf{0} & \color{AccentRed}\textbf{118.2} \\
\bottomrule
\end{tabular}
\end{table}

\vskip0.3cm
{\color{MahidolBlue}$\Rightarrow$} EADD: \textbf{zero false alarms} \quad D3: up to 87.4 on autocorrelated data\\[4pt]
Mann--Whitney $p = 0.0101$ (statistically significant difference)
\end{block}

\vskip1cm

% --- Summary ---
\begin{alertblock}{Summary of Results}
\large
\renewcommand{\arraystretch}{1.4}
\begin{center}
\begin{tabular}{lcc}
\toprule
\textbf{Metric} & \textbf{EADD} & \textbf{D3} \\
\midrule
Temporal coverage     & \textbf{4/4}  & 2/4 \\
Real-world MDR        & \textbf{0\%}  & 7.7\% \\
Avg.\ MTD (samples)   & \textbf{1{,}661} & 1{,}725 \\
Feature attribution   & \textbf{3/3}  & --- \\
False alarm total     & \textbf{0}    & 118.2 \\
\bottomrule
\end{tabular}
\end{center}
\end{alertblock}

\vskip1cm

% --- Conclusion ---
\begin{block}{7. Conclusions \& Future Work}
\begin{itemize}
  \item EADD transforms drift detection from a \textbf{binary alarm} into an \textbf{actionable diagnostic tool} with feature-level attribution and remediation prescriptions.
  \item \textbf{Strengths:} 100\% temporal coverage, zero false alarms, correct feature identification in all tested scenarios, competitive detection speed.
  \item \textbf{Limitations:} Computational overhead of retraining LightGBM per monitoring cycle; prescription rules are currently heuristic.
  \item \textbf{Future directions:}
  \begin{itemize}
    \item Online incremental discriminator updates
    \item Feedback-loop integration with CI/CD retraining
    \item Extension to multi-modal data streams
  \end{itemize}
\end{itemize}
\end{block}

\vskip1cm

% --- References ---
\begin{block}{References}
\small
\begin{enumerate}
  \item J.~Lu et al., ``Learning under concept drift: A review,'' \textit{IEEE TKDE}, 2018.
  \item O.~G\"{o}z\"{u}a\c{c}{\i}k et al., ``Unsupervised concept drift detection with a discriminative classifier,'' \textit{ACM CIKM}, 2019.
  \item B.~Lukats et al., ``Unsupervised concept drift detection from deep learning representations in real-time,'' \textit{CIKM}, 2025.
  \item S.~M.~Lundberg and S.-I.~Lee, ``A unified approach to interpreting model predictions,'' \textit{NeurIPS}, 2017.
  \item G.~Ke et al., ``LightGBM: A highly efficient gradient boosting decision tree,'' \textit{NeurIPS}, 2017.
\end{enumerate}
\end{block}

\end{column}
\end{columns}

\end{frame}
\end{document}
