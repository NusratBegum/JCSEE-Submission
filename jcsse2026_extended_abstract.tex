% JCSSE Bangkok 2026 - Extended Abstract Template (Poster Session)
% The 23rd International Joint Conference on Computer Science and Software Engineering
% 24-27 June 2026, Bangkok, Thailand
% https://jcsse2026.org/
%
% IEEE Conference Proceedings format - A4 paper, Times New Roman 10pt
% Extended Abstract: 1-2 pages | Full Paper: up to 6 pages
% BLIND REVIEW: Do NOT include author names/affiliations until camera-ready

\documentclass[10pt,twocolumn,a4paper]{article}

% ===== PACKAGES =====
\usepackage[utf8]{inputenc}
\usepackage[T1]{fontenc}
\usepackage{times}                      % Times New Roman font
\usepackage{mathptmx}                   % Times for math
\usepackage{graphicx}                   % For figures
\usepackage{amsmath,amssymb}            % Math support
\usepackage{caption}                    % Caption styling
\usepackage{array}                      % Table improvements
\usepackage{booktabs}                   % Professional tables
\usepackage{enumitem}                   % List customization
\usepackage{hyperref}                   % Hyperlinks (disabled for IEEE style)
\usepackage{xcolor}                     % Color support
\usepackage{balance}                    % Balance last page columns
\usepackage{titlesec}                   % Section formatting
\usepackage{fancyhdr}                   % Header/footer control
\usepackage{flushend}                   % Balance final page columns

% ===== PAGE LAYOUT (A4 Paper - IEEE Standard) =====
% A4: 210mm x 297mm
% Top margin: 19mm, Bottom: 43mm, Left/Right: 13mm
% Column width: 88mm, Column gap: 6mm
\usepackage[
    a4paper,
    top=19mm,
    bottom=43mm,
    left=13mm,
    right=13mm,
    columnsep=6mm
]{geometry}

% ===== NO PAGE NUMBERS =====
\pagestyle{empty}
\fancyhf{}
\renewcommand{\headrulewidth}{0pt}
\renewcommand{\footrulewidth}{0pt}

% ===== HYPERREF SETUP (disable visible links per IEEE style) =====
\hypersetup{
    colorlinks=false,
    pdfborder={0 0 0},
    hidelinks
}

% ===== SECTION FORMATTING =====
% First-order headings: 10pt small caps, centered, 12pt before, 6pt after
\titleformat{\section}
    {\normalfont\normalsize\scshape\centering}
    {}
    {0pt}
    {}
\titlespacing*{\section}{0pt}{12pt}{6pt}

% Second-order headings: 10pt italic, flush left, Roman numerals, 9pt before, 3pt after
\titleformat{\subsection}
    {\normalfont\normalsize\itshape}
    {\Roman{subsection}.}
    {0.5em}
    {}
\titlespacing*{\subsection}{0pt}{9pt}{3pt}

% Reset subsection numbering within sections
\renewcommand{\thesubsection}{\Roman{subsection}}

% ===== CAPTION FORMATTING =====
% 10pt Times, small caps, left-justified
\captionsetup{
    font=small,
    labelfont={sc},
    justification=justified,
    singlelinecheck=false,
    skip=6pt
}
\captionsetup[figure]{position=below}
\captionsetup[table]{position=above}

% ===== PARAGRAPH FORMATTING =====
\setlength{\parindent}{1pc}             % 1 pica indent (~0.17 inch)
\setlength{\parskip}{0pt}               % No space between paragraphs

% ===== LIST FORMATTING =====
\setlist[itemize]{
    leftmargin=0.5in,
    topsep=0pt,
    partopsep=0pt,
    parsep=0pt,
    itemsep=0pt
}
\setlist[enumerate]{
    leftmargin=0.5in,
    topsep=0pt,
    partopsep=0pt,
    parsep=0pt,
    itemsep=0pt
}

% ===== CUSTOM COMMANDS =====
% Title formatting (24pt, centered)
\newcommand{\papertitle}[1]{%
    \twocolumn[%
        \begin{center}
            {\fontsize{24pt}{28pt}\selectfont #1 \par}
            \vspace{12pt}
        \end{center}
    ]
}

% Author block
\newcommand{\authorblock}[3]{%
    \begin{tabular}{c}
        \textbf{#1} \\
        \textit{#2} \\
        #3
    \end{tabular}
}

% Abstract environment (10pt bold italic)
\renewenvironment{abstract}{%
    \noindent\textbf{\textit{Abstract}}\textemdash\textbf{\textit{}}%
}{%
    \par\vspace{6pt}
}

% Index Terms
\newcommand{\indexterms}[1]{%
    \noindent\textbf{\textit{Index Terms}}\textemdash\textit{#1}%
    \par\vspace{12pt}
}

% Source for figures/tables
\newcommand{\source}[1]{%
    \vspace{1pt}
    {\footnotesize Source: #1}
}

% ===== DOCUMENT =====
\begin{document}

% ===== TITLE =====
\twocolumn[
\begin{center}
    {\fontsize{24pt}{28pt}\selectfont Your Paper Title Here: Use Title Case Capitalization\par}
    \vspace{18pt}
    
    % ===== BLIND REVIEW VERSION =====
    % IMPORTANT: For initial submission, DO NOT include author information.
    % Author names, affiliations, and email addresses should only be added
    % in the final camera-ready version after paper acceptance.
    %
    % Uncomment the author block below ONLY for camera-ready submission:
    % -----------------------------------------------------------------------
    % \begin{tabular}{ccc}
    %     \begin{tabular}{c}
    %         \textbf{First Author Name} \\
    %         \textit{Affiliation} \\
    %         author1@email.com
    %     \end{tabular}
    %     &
    %     \begin{tabular}{c}
    %         \textbf{Second Author Name} \\
    %         \textit{Affiliation} \\
    %         author2@email.com
    %     \end{tabular}
    %     &
    %     \begin{tabular}{c}
    %         \textbf{Third Author Name} \\
    %         \textit{Affiliation} \\
    %         author3@email.com
    %     \end{tabular}
    % \end{tabular}
    % \vspace{18pt}
    % -----------------------------------------------------------------------
    
\end{center}
]

% ===== ABSTRACT (~150 words) =====
\noindent\textbf{\textit{Abstract}}\textemdash\textit{Provide a concise summary of approximately 150 words. The abstract should clearly present the motivation for your research, the methodology employed, and the key findings or contributions of your work. Avoid using abbreviations, acronyms, or citations in the abstract. The abstract serves as a standalone summary that will appear in conference proceedings and digital libraries. Write in a clear, objective style that allows readers to quickly understand the significance and scope of your research. This is placeholder text to demonstrate the approximate length expected---replace this entire paragraph with your actual abstract content.}

\vspace{6pt}

% ===== INDEX TERMS =====
\noindent\textbf{\textit{Index Terms}}\textemdash\textit{First keyword, second keyword, third keyword, fourth keyword.}

\vspace{12pt}

% ===== INTRODUCTION =====
\section{Introduction}

This \LaTeX{} template is prepared for JCSSE 2026---The 23rd International Joint Conference on Computer Science and Software Engineering, to be held June 24--27, 2026 in Bangkok, Thailand. For extended abstracts (poster session), submissions must be limited to 1--2 pages. For full papers (regular/special sessions), submissions must not exceed 6 pages.

All submissions must follow the IEEE Conference Proceedings format with A4 paper size. For blind review, author names and affiliations must be omitted. This template is pre-configured for the correct format---simply replace the placeholder content with your research.

% ===== MAIN SECTIONS =====
\section{Related Work}

\subsection{Literature Review}

Include your literature review here. Type your main text in 10-point Times, single-spaced. All paragraphs should be indented approximately 1/6-inch. Be sure your text is fully justified---that is, flush left and flush right.

When citing references, use square brackets. For example, previous research has shown significant results in this area~[1]. Multiple citations can be combined~[2], [3].

\subsection{Research Gap}

Describe the research gap your work addresses. Use proper formatting for any special terms or variables.

% ===== METHODOLOGY =====
\section{Methodology}

Describe your research methodology here. You can include equations as needed. Number equations consecutively with equation numbers in parentheses flush with the right margin:

\begin{equation}
    a + b = c
\end{equation}

Be sure that symbols in your equation have been defined before the equation appears or immediately following. Use ``Equation~(1)'' at the beginning of a sentence, but otherwise use ``(1)'' to reference equations.

% ===== RESULTS =====
\section{Results and Discussion}

Present your results here. You can include figures and tables as needed.

% Example figure
\begin{figure}[htbp]
    \centering
    % \includegraphics[width=\columnwidth]{your_figure.png}
    \framebox[\columnwidth]{\rule{0pt}{1.5in}} % Placeholder - replace with actual figure
    \caption{Caption style for describing figures. Provide a short title followed by a detailed explanation of the figure content.}
    \label{fig:example}
\end{figure}

As shown in Figure~\ref{fig:example}, the results demonstrate... Continue your discussion here.

% Example table
\begin{table}[htbp]
    \caption{Table Type Styles}
    \label{tab:example}
    \centering
    \begin{tabular}{|l|c|c|}
        \hline
        \textbf{Table Head} & \textbf{Column Head} & \textbf{Column Head} \\
        \hline
        Table column subhead & Subhead & Subhead \\
        \hline
        Table copy & More table copy & Data \\
        \hline
    \end{tabular}
\end{table}

Table~\ref{tab:example} presents the comparative results. Discuss your findings in detail here.

% ===== CONCLUSION =====
\section{Conclusion}

Summarize your key findings and their implications. State the main contributions of your work and suggest directions for future research.

% ===== ACKNOWLEDGEMENTS (Optional) =====
% IMPORTANT: Comment out for blind review. Uncomment ONLY for camera-ready.
% -----------------------------------------------------------------------
% \section*{Acknowledgements}
%
% If you wish to identify funding sources or significant contributions by 
% others, include your acknowledgements here. Only include this section 
% in the final camera-ready copy after paper acceptance.
% -----------------------------------------------------------------------

% ===== REFERENCES =====
% 9-point Times, single-spaced
\section*{References}
\begin{small}
\begin{enumerate}[label={[\arabic*]}, leftmargin=*, itemsep=1pt, parsep=0pt]
    \item I. Thompson, ``Women and feminism in technical communication,'' \textit{J. Bus. Tech. Commun.}, vol.~13, no.~2, pp.~154--178, 1999.
    
    \item M.~S. MacNealy, \textit{Strategies for Empirical Research in Writing}. Boston, MA: Allyn and Bacon, 1999.
    
    \item J.~H. Watt and S.~A. van den Berg, \textit{Research Methods for Communication Science}. Boston, MA: Allyn and Bacon, 1995.
    
    \item S. Kleinmann, ``The reciprocal relationship of workplace culture and review,'' in \textit{Writing in the Workplace: New Research Perspectives}, R.~Spilka, Ed. Carbondale, IL: Southern Illinois University Press, 1993, pp.~56--70.
    
    \item K.~St. Amant, ``Virtual office communication protocols: A system for managing international virtual teams,'' in \textit{Proc. IEEE Int. Prof. Commun. Conf.}, 2005, pp.~703--717.
    
    \item Structural Engineering Society--International. [Online]. Available: http://www.seaint.org
    
    \item M. Tohidi et al., ``Getting the right design and the design right: Testing many is better than one,'' in \textit{Proc. ACM-SIGCHI Conf. on Human Factors in Computing Syst. (CHI'06)}, 2006, pp.~1243--1252.
\end{enumerate}
\end{small}

% ===== ABOUT THE AUTHORS =====
% IMPORTANT: Comment out this entire section for blind review submission.
% Uncomment ONLY for final camera-ready version after acceptance.
% -----------------------------------------------------------------------
% \section*{About the Authors}
%
% \textbf{First Author Name} is a Graduate Student at [University/Institution]. 
% Research interests include [list interests]. Contact: author1@email.com
%
% \vspace{6pt}
%
% \textbf{Second Author Name} is [position] at [University/Institution]. 
% Research focus includes [list focus areas]. Contact: author2@email.com
%
% \vspace{6pt}
%
% \textbf{Third Author Name} is [position] at [University/Institution]. 
% Specializes in [list specialization]. Contact: author3@email.com
% -----------------------------------------------------------------------

% Balance columns on last page
\balance

\end{document}
